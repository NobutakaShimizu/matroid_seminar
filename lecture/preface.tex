\chapter*{序文}
\emph{理論計算機科学 (theoretical computer science)} とは計算機の能力や限界に迫る分野であり, いわゆる応用数学の一つだが, 
    実は純粋数学(特に離散数学)の幅広い分野の概念が登場するいわば「交差点」となるような分野である.
特に近年はグラフ理論におけるエクスパンダー性と呼ばれる性質を単体複体に自然に拡張した
    \emph{高次元エクスパンダー}と呼ばれる概念が近年の理論計算機科学の大きな潮流となっている.

本講義ではまず理論計算機科学においてスタンダードなランダムウォークの定義に基づいてグラフや単体複体上のランダムウォークの理論を構築していき, 高次元エクスパンダーの定義を与え,
Kaufman--Oppenheimの定理やOppenheimのトリクルダウン定理の証明を与える.
そしてマトロイドと呼ばれる離散構造のエクスパンダー性を持つことを証明し, マトロイド上の基交換ウォークの混交性という長年の未解決問題(Mihail--Vazirani予想)の\citet{ALOV24}による証明を与える.

なお, 本講義資料作成にあたって
見村 万佐人先生,
林 興養先生,
来嶋 秀治先生,
平原 秀一先生
に様々な誤植やコメントをいただきました.
この場を借りてお礼申し上げます.
